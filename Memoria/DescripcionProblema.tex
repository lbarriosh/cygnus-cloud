
\chapter{Descripci�n del problema}


\section{Introducci�n}

El uso de las tecnolog�as de la informaci�n y la comunicaci�n (TIC)
en el sector educativo no es algo nuevo. Ya en la d�cada de los 60,
Plattner y Herron reconocieron que el uso de computadores era extremadamente
�til para ense�ar a tomar decisiones en situaciones reales y a desempe�ar
distintos cargos, y tambi�n para fomentar la participaci�n. Por ello,
no es de extra�ar que las TIC hayan adquirido un papel protagonista
en el panorama educativo actual.

De esta manera, a medida que las TIC han ido implant�ndose en la sociedad,
su uso en la universidad no ha hecho m�s que crecer. Adem�s, la puesta
en funcionamiento del llamado Espacio Europeo de Educaci�n Superior
ha hecho que las sesiones pr�cticas tengan un importante peso en todos
los planes de estudio universitarios. Y dado que en estas sesiones
pr�cticas se hace uso de las TIC, podemos afirmar que los estudiantes
pasar�n una cantidad de tiempo considerable delante de un ordenador
personal durante su estancia en la universidad.

La Universidad Complutense de Madrid (UCM) es la universidad presencial
espa�ola con mayor n�mero de alumnos. Para que todos ellos puedan
realizar sus trabajos acad�micos, existe un gran n�mero de aulas de
inform�tica y laboratorios repartidos entre sus diversos centros. 

Naturalmente, la gesti�n y el mantenimiento de estos espacios es responsabilidad
de personal de administraci�n y servicios, dependiente, en �ltima
instancia, de los Servicios Inform�ticos de la universidad. 

A lo largo de este cap�tulo, presentaremos el modelo de gesti�n de
los laboratorios de la Facultad de Inform�tica de la UCM, para posteriormente
discutir sus inconvenientes y presentar las principales caracter�sticas
del sistema \emph{Cygnus Cloud}.




